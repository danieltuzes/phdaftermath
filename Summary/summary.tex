%!TEX root = ../thesis.tex
%*******************************************************************************
%****************************** Seventh Chapter **********************************
%*******************************************************************************
\chapter[Summary, összefoglaló]{Summary, összefoglaló} \label{chapter:summary}
\section*{In English}
The topic of my doctoral thesis was inspired by the diversity of the collective and stochastic properties of dislocations. Nowadays the size scales accessible for computational simulations and experiments overlap for crystalline materials. They can be modelled with computers and can be investigated by, for example, scanning electron microscopes, even in an in-situ deformation setup too.

As the first step of my thesis I approached the issue from a theoretical point of view. I developed a cellular automaton model based on the continuum theory of dislocations and investigated the role of the relevant parameters in the region of small deformations. The model handles the flow stress on cell-level ($\tau_w$) and it is considered as a random variable and calibrated via lower scale, discrete dislocation dynamic (DDD) simulations. The expected value of $\tau_w$ and the size of the applied discrete plastic strain on the cells are calibrated via the comparison of the stress-strain curve of the CA model and two other DDD models. The efficiency of the multiscale modelling is reflected on that that beyond the fitted properties of the model it shows the same type of universality classes with the DDD simulations. Based on the findings a plasticity model has been also introduced.

The CA model is applicable on materials with internal disorder that undergo strain softening, if the fracture is due to strain localisation. To model this behaviour a softening mechanism has been introduced. The disorder in the material is provided by the distribution of $\tau_w$. The results of the simulations show, that in materials with higher disorder the applicable highest stress is significantly larger, and failure occurs at considerably larger plastic strain.

The CA models used above have strong assumptions on dislocation correlations. To this end the equation of motions of the CA model is extended with stress terms attributable to these correlations. Linear stability analysis (LSA) shows the possibility of dislocation pattern formation and link the free parameters of the model to the characteristic wavelength of the pattern. My simulation results obtained from this CA model follows well the prediction of the LSA, so does a non-discrete model, where in contrary to the extremal dynamics of the CA, hydrodynamic-like transport equations are used. Despite the elemental differences between the models they both built upon new stress terms attributed to the more precise continuum theory, and therefore they indeed show qualitatively same results, showing the robustness of the theory.

In the last part of my thesis I approached the topic from an experimental point of view. In the micron scale the plastic deformation of crystalline materials show avalanche-like behaviour. The key of the research work is the recognition, that in this size scale the deformation response of each micron-sized sample is different from sample to sample, therefore, characterisation of materials must rely on a statistical approach. This requires a large amount of data measured on the samples, and the origin of the data measured must be well interpreted and evaluated. To this end, on the one hand, a new micropillar-fabrication method is proposed facilitating the mass production of the samples with the required shape. On the other hand, a unique experimental setup has been implemented making it possible to track the in-situ compression procedure faster than ever before in an electron microscope, and due to an attached acoustic emission detector coupling the avalanches observed with the acoustic signals detected became possible.

\section*{Magyarul}
Doktori tézisem témáját a diszlokációk mozgásának kollektív és sztochasztikus tulajdonságainak a sokszínűsége biztosította. Ma már a vizsgált kristályos anyagok méretskálái összeérnek a numerikus szimulációs és a kísérleti oldalról: olyan anyagok modellezhetőek számítógéppel, amelyek vizsgálhatóak pl. pásztázó elektronmikroszkópban, akár in-situ deformáció közben is.

Doktori munkám első lépéseként elméleti úton közelítettem meg a témakört. A diszlokációk kontinuum-elméleti leírására épülő sejtautomata (CA) modellt fejlesztettem és vizsgáltam a releváns paraméterek szerepét a kis deformációk tartományában. A CA modellben a cellaszintű folyáshatárt ($\tau_w$-t) véletlen változóként kezeltem, és alacsonyabb skálájú, diszkrét diszlokációdinamikai modellek alapján kalibráltam. A $\tau_w$ várható értékét és a cellákban alkalmazott diszkrét deformációs lépés nagyságát pedig a CA, és két másik DDD modell feszültség-deformációs görbéi alapján kalibráltam. A CA modell eredményessége abban mutatkozik meg, hogy az illesztett tulajdonságokon felül a modell azonos típusú univerzális skálázási tulajdonságokat mutat, mint a DDD szimulációk. Az eredmények alapján egy plaszticitás modell is bemutatásra került.

A CA modell alkalmazható olyan alakítási lágyulást szenvedő anyagok törésének modellezésére is, ahol a törést a deformáció lokalizáció okozza. Ehhez bevezettem a modellben egy lágyulási mechanizmust. Az így kapott modellben a rendszer rendezetlenségét a $\tau_w$ eloszlása biztosítja. A szimulációk eredménye azt mutatta, hogy a nagyobb rendezetlenségű anyagokban az alkalmazható maximális feszültség jelentősen nő és jóval nagyobb plasztikus deformáció után következik be a törés.

A fentebb használt CA modell erős feltételezéseket tesz a diszlokáció korrelációkra. Ezért a CA modell mozgásegyenleteit kiegészítettem olyan további feszültségtagokkal, amelyek eredete a korrelációkra vezethető vissza. Lineáris stabilitásanalízis (LSA) alapján tudható, hogy lehetőség van diszlokációmintázatok fejlődésére, valamint az LSA kapcsolatot teremt a modell szabad paraméterei és a mintázat jellemző hullámhossza között. A CA szimulációs eredményei azt mutatták, hogy a modell jól követi az LSA jóslatát, hasonlóan egy nem diszkrét, és a CA modell extrém dinamikájával szemben egy hidrodinamika-szerű transzport-egyenletet használó modellel. Ez a kétfajta megvalósítás alapvetően különbözik egymástól, de ugyanazokon, a pontosabb kontinuumelmélethez tartozó új erőkön alapulnak, és eredményeink alapján éppen ezért minőségileg azonos eredményt is adnak, amely mutatja az elmélet robusztusságát.

A doktori munkámat a téma kísérleti oldalról való megközelítésével zártam. Itt a mikron méretű fémes anyagok plasztikus deformációjában megjelenő lavinaszerű viselkedés vizsgálata volt az előtérben. A munka kulcseleme, hogy ezen a méretskálán minden egyes mikroméretű kristály deformációs válasza más és más, így az anyagoknak csak statisztikai értelemben adhatunk tulajdonságokat. Ehhez elengedhetetlen, hogy nagymennyiségű adatot lehessen a mintákról összegyűjteni, és hogy a mért adatok fizikai háttere világos legyen. Ennek elősegítése céljából egyrészt egy új mikrooszlop-megmunkálási eljárást javasoltunk, amellyel minden eddiginél gyorsabban lehet a kívánt alakú mikrooszlopokat előállítani. Másrészt egy unikális kísérleti összeállítást valósítottunk meg, amelynek segítségével minden eddiginél gyorsabb lekövetéssel lehetséges a mikrooszlopok összenyomásának in-situ elektronmikroszkópos vizsgálata, így egy csatolt akusztikus emissziós detektorral pontosan párosítani tudjuk a diszlokáció lavinákat és az érzékelt akusztikus jeleket.
